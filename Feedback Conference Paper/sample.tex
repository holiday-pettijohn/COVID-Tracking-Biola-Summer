% This is a comment

% Everything from the top of the file to the \begin{document} is
% called the preamble.  It contains information used to set up 
% the document.

\documentclass{article}  
% This has to be the first non comment line of the document.  It tells you what kind
% of document it is.  options: book, article, letter, memo, report, ...
% The following line is needed if you want to read in postscript figures
\usepackage{epsfig}  

% set various sizes:
\setlength{\textwidth}{6.5in}
\setlength{\textheight}{8.7in}
\setlength{\voffset}{-.7in}
\setlength{\oddsidemargin}{0in}

\begin{document}  % This is where the document starts.

\title{A Simple Example Using \LaTeX}   % \LaTeX makes a fancy LaTeX logo.
\author{R. J. LeVeque and T. P. Chartier}
\date{\today}
\maketitle

\abstract{This short paper illustrates basic techniques in the 
use of the \LaTeX~ text formatting language, which is useful in typesetting  
scientific and mathematical papers. This paper can serve as a short tutorial and
as a useful template for a beginning \LaTeX~ programmer.}
% ~ denotes a forced space.  Remove the ~ appended to the end of \LaTeX and see what happens.
% Latex is also case sensitive.  If you type \LaTex, you will receive a compiling error.

\section{Introduction}

This paper illustrates how various typesetting is accomplished in \LaTeX.
You are encouraged to view both the source code ({\tt sample.tex}) 
and the typeset paper ({\tt sample.dvi}). 

First, notice that components of your paper such as {\it sections, subsections,} and 
{\it equations} are numbered automatically.  

%\subsection{Lists}
%Making an itemized list is easy:
%\begin{itemize}
%\item If the \LaTeX~ file is called {\tt sample.tex} then output is created by
%executing \\
%   {\tt latex sample} \\
%This creates a file {\tt sample.dvi}.
%To view or print, see documentation for your system.
%\item Note that leaving one or more blank lines
%     in the input
%gives the start of a new paragraph.  Otherwise blanks
%do not   matter.  % and this is a comment that won't print.
%To force a line to end, use double backslash as in the previous item as seen in {\tt sample.tex}.
%
%\item Anything after a \% is a comment that won't appear in the output.
%   % so you need to use the special macro \% if you want % to show up!!
%\end{itemize}
%
%\section{Mathematical formulas}
%
%% This is a citation.  Look at the bibliography to see the reference!
%This paper supplies only a taste of how mathematical formulas are created in \LaTeX.  
%For more information, see a \LaTeX~ manual such as \cite{go-mi-sa:latex}.  
%As a short introduction, a few pointers are contained in this section. 
%
%Dollar signs places \LaTeX~ into its math mode, which is where mathematical formulas and equations
%are created.  To include an equation in a current line of text, place one dollar sign (\$) before 
%and after the equation, for example we might define $f(x) = 3e^x$.   If an equation should be displayed,
%place two dollar signs (\$\$) before and after the equation.  For example, we might define:
%$$
%\sum_{j=1}^N j = \frac{N(N+1)}{2}.
%$$
%If the equation should be numbered, use the following notation (see the \LaTeX~ file):
%\begin{equation}  
%e^{i\pi} = \cos(\pi) = -1.
%\label{cosequation}
%\end{equation}
%This equation was given a label (which is optional).  Hence, the equation
%can be referred to later as (\ref{cosequation}) rather than hardcoding such numbers into
%the \LaTeX~ file.  Therefore, adding new equations does not facilitate a need for the programmer 
%to manually renumber the equations. 
%(Note that you will have to run \LaTeX~ twice on your {\tt .tex}
%file for numbering to appear properly.  The first run stores the labels in
%the file {\tt sample.aux} and the second time run reads these labels at the
%beginning of processing.)
%
%\section{Matrices}
%
%Matrices can be made using an ``array''.  Here's a useful definition that 
%makes it easier to define matrices:  (see the \LaTeX~ source file)
%\newenvironment{mat}{\left[\begin{array}{ccccccccccccc}}{\end{array}\right]}
%
%Here's a simple matrix equation using this definition:
%\begin{equation}
%\begin{mat} a_{11} & a_{12} \\ a_{21} & a_{22} \end{mat}
%\begin{mat} x_1\\ x_2 \end{mat}
%= \begin{mat} b_1\\ b_2 \end{mat}
%\label{matrixequation}
%\end{equation}
%
%\section{Figures}
%Many plotting packages (e.g. matlab) allow you to produce an encapsulated
%postscript file (ending in {\tt .eps}).   Then the graphic can be incorporated
%into a paper with the following command (see the \LaTeX~ file)
%
%%\begin{figure}
%%\hfil\epsfig{file=fig.eps, width=4in}\hfil
%%\caption{\label{figlabel} Histogram of 1000 normally distributed random
%%numbers.}
%%\end{figure}
%
%Note that the {\tt caption} command gives the figure a number that can be referred
%%to later as Figure \ref{figlabel}.
%
%The file {\tt fig.eps} was created in matlab with the commands:
%\begin{verbatim}
%     >> r = randn(1000,1);
%     >> hist(r)             
%     >> colormap([.7 .7 .7])    % to change the color and make it print better
%     >> print fig.eps
%\end{verbatim}
%
%

\section{Bibliography and citations}
See a \LaTeX~ manual such as \cite{go-mi-sa:latex}
for complete information on the use of a bibliography and citations.  The 



\bibliographystyle{plain}   % tells how to format bibliographic entries
\bibliography{c19References}    % tells that the database is in samplebib.bib
\end{document} % This is the end of the document
